% Options for packages loaded elsewhere
\PassOptionsToPackage{unicode}{hyperref}
\PassOptionsToPackage{hyphens}{url}
%
\documentclass[
  ignorenonframetext,
]{beamer}
\title{RM\_LBB}
\author{Johan Setiawan}
\date{12/7/2021}

\usepackage{pgfpages}
\setbeamertemplate{caption}[numbered]
\setbeamertemplate{caption label separator}{: }
\setbeamercolor{caption name}{fg=normal text.fg}
\beamertemplatenavigationsymbolsempty
% Prevent slide breaks in the middle of a paragraph
\widowpenalties 1 10000
\raggedbottom
\setbeamertemplate{part page}{
  \centering
  \begin{beamercolorbox}[sep=16pt,center]{part title}
    \usebeamerfont{part title}\insertpart\par
  \end{beamercolorbox}
}
\setbeamertemplate{section page}{
  \centering
  \begin{beamercolorbox}[sep=12pt,center]{part title}
    \usebeamerfont{section title}\insertsection\par
  \end{beamercolorbox}
}
\setbeamertemplate{subsection page}{
  \centering
  \begin{beamercolorbox}[sep=8pt,center]{part title}
    \usebeamerfont{subsection title}\insertsubsection\par
  \end{beamercolorbox}
}
\AtBeginPart{
  \frame{\partpage}
}
\AtBeginSection{
  \ifbibliography
  \else
    \frame{\sectionpage}
  \fi
}
\AtBeginSubsection{
  \frame{\subsectionpage}
}
\usepackage{amsmath,amssymb}
\usepackage{lmodern}
\usepackage{iftex}
\ifPDFTeX
  \usepackage[T1]{fontenc}
  \usepackage[utf8]{inputenc}
  \usepackage{textcomp} % provide euro and other symbols
\else % if luatex or xetex
  \usepackage{unicode-math}
  \defaultfontfeatures{Scale=MatchLowercase}
  \defaultfontfeatures[\rmfamily]{Ligatures=TeX,Scale=1}
\fi
\usetheme[]{AnnArbor}
% Use upquote if available, for straight quotes in verbatim environments
\IfFileExists{upquote.sty}{\usepackage{upquote}}{}
\IfFileExists{microtype.sty}{% use microtype if available
  \usepackage[]{microtype}
  \UseMicrotypeSet[protrusion]{basicmath} % disable protrusion for tt fonts
}{}
\makeatletter
\@ifundefined{KOMAClassName}{% if non-KOMA class
  \IfFileExists{parskip.sty}{%
    \usepackage{parskip}
  }{% else
    \setlength{\parindent}{0pt}
    \setlength{\parskip}{6pt plus 2pt minus 1pt}}
}{% if KOMA class
  \KOMAoptions{parskip=half}}
\makeatother
\usepackage{xcolor}
\IfFileExists{xurl.sty}{\usepackage{xurl}}{} % add URL line breaks if available
\IfFileExists{bookmark.sty}{\usepackage{bookmark}}{\usepackage{hyperref}}
\hypersetup{
  pdftitle={RM\_LBB},
  pdfauthor={Johan Setiawan},
  hidelinks,
  pdfcreator={LaTeX via pandoc}}
\urlstyle{same} % disable monospaced font for URLs
\newif\ifbibliography
\usepackage{color}
\usepackage{fancyvrb}
\newcommand{\VerbBar}{|}
\newcommand{\VERB}{\Verb[commandchars=\\\{\}]}
\DefineVerbatimEnvironment{Highlighting}{Verbatim}{commandchars=\\\{\}}
% Add ',fontsize=\small' for more characters per line
\usepackage{framed}
\definecolor{shadecolor}{RGB}{248,248,248}
\newenvironment{Shaded}{\begin{snugshade}}{\end{snugshade}}
\newcommand{\AlertTok}[1]{\textcolor[rgb]{0.94,0.16,0.16}{#1}}
\newcommand{\AnnotationTok}[1]{\textcolor[rgb]{0.56,0.35,0.01}{\textbf{\textit{#1}}}}
\newcommand{\AttributeTok}[1]{\textcolor[rgb]{0.77,0.63,0.00}{#1}}
\newcommand{\BaseNTok}[1]{\textcolor[rgb]{0.00,0.00,0.81}{#1}}
\newcommand{\BuiltInTok}[1]{#1}
\newcommand{\CharTok}[1]{\textcolor[rgb]{0.31,0.60,0.02}{#1}}
\newcommand{\CommentTok}[1]{\textcolor[rgb]{0.56,0.35,0.01}{\textit{#1}}}
\newcommand{\CommentVarTok}[1]{\textcolor[rgb]{0.56,0.35,0.01}{\textbf{\textit{#1}}}}
\newcommand{\ConstantTok}[1]{\textcolor[rgb]{0.00,0.00,0.00}{#1}}
\newcommand{\ControlFlowTok}[1]{\textcolor[rgb]{0.13,0.29,0.53}{\textbf{#1}}}
\newcommand{\DataTypeTok}[1]{\textcolor[rgb]{0.13,0.29,0.53}{#1}}
\newcommand{\DecValTok}[1]{\textcolor[rgb]{0.00,0.00,0.81}{#1}}
\newcommand{\DocumentationTok}[1]{\textcolor[rgb]{0.56,0.35,0.01}{\textbf{\textit{#1}}}}
\newcommand{\ErrorTok}[1]{\textcolor[rgb]{0.64,0.00,0.00}{\textbf{#1}}}
\newcommand{\ExtensionTok}[1]{#1}
\newcommand{\FloatTok}[1]{\textcolor[rgb]{0.00,0.00,0.81}{#1}}
\newcommand{\FunctionTok}[1]{\textcolor[rgb]{0.00,0.00,0.00}{#1}}
\newcommand{\ImportTok}[1]{#1}
\newcommand{\InformationTok}[1]{\textcolor[rgb]{0.56,0.35,0.01}{\textbf{\textit{#1}}}}
\newcommand{\KeywordTok}[1]{\textcolor[rgb]{0.13,0.29,0.53}{\textbf{#1}}}
\newcommand{\NormalTok}[1]{#1}
\newcommand{\OperatorTok}[1]{\textcolor[rgb]{0.81,0.36,0.00}{\textbf{#1}}}
\newcommand{\OtherTok}[1]{\textcolor[rgb]{0.56,0.35,0.01}{#1}}
\newcommand{\PreprocessorTok}[1]{\textcolor[rgb]{0.56,0.35,0.01}{\textit{#1}}}
\newcommand{\RegionMarkerTok}[1]{#1}
\newcommand{\SpecialCharTok}[1]{\textcolor[rgb]{0.00,0.00,0.00}{#1}}
\newcommand{\SpecialStringTok}[1]{\textcolor[rgb]{0.31,0.60,0.02}{#1}}
\newcommand{\StringTok}[1]{\textcolor[rgb]{0.31,0.60,0.02}{#1}}
\newcommand{\VariableTok}[1]{\textcolor[rgb]{0.00,0.00,0.00}{#1}}
\newcommand{\VerbatimStringTok}[1]{\textcolor[rgb]{0.31,0.60,0.02}{#1}}
\newcommand{\WarningTok}[1]{\textcolor[rgb]{0.56,0.35,0.01}{\textbf{\textit{#1}}}}
\usepackage{graphicx}
\makeatletter
\def\maxwidth{\ifdim\Gin@nat@width>\linewidth\linewidth\else\Gin@nat@width\fi}
\def\maxheight{\ifdim\Gin@nat@height>\textheight\textheight\else\Gin@nat@height\fi}
\makeatother
% Scale images if necessary, so that they will not overflow the page
% margins by default, and it is still possible to overwrite the defaults
% using explicit options in \includegraphics[width, height, ...]{}
\setkeys{Gin}{width=\maxwidth,height=\maxheight,keepaspectratio}
% Set default figure placement to htbp
\makeatletter
\def\fps@figure{htbp}
\makeatother
\setlength{\emergencystretch}{3em} % prevent overfull lines
\providecommand{\tightlist}{%
  \setlength{\itemsep}{0pt}\setlength{\parskip}{0pt}}
\setcounter{secnumdepth}{-\maxdimen} % remove section numbering
\ifLuaTeX
  \usepackage{selnolig}  % disable illegal ligatures
\fi

\begin{document}
\frame{\titlepage}

\begin{frame}[fragile]
\begin{Shaded}
\begin{Highlighting}[]
\FunctionTok{library}\NormalTok{(dplyr)}
\end{Highlighting}
\end{Shaded}

\begin{verbatim}
## 
## Attaching package: 'dplyr'
\end{verbatim}

\begin{verbatim}
## The following objects are masked from 'package:stats':
## 
##     filter, lag
\end{verbatim}

\begin{verbatim}
## The following objects are masked from 'package:base':
## 
##     intersect, setdiff, setequal, union
\end{verbatim}

\begin{Shaded}
\begin{Highlighting}[]
\FunctionTok{library}\NormalTok{(ggplot2)}
\FunctionTok{library}\NormalTok{(GGally)}
\end{Highlighting}
\end{Shaded}

\begin{verbatim}
## Registered S3 method overwritten by 'GGally':
##   method from   
##   +.gg   ggplot2
\end{verbatim}

\begin{Shaded}
\begin{Highlighting}[]
\FunctionTok{library}\NormalTok{(performance)}
\FunctionTok{library}\NormalTok{(MLmetrics)}
\end{Highlighting}
\end{Shaded}

\begin{verbatim}
## 
## Attaching package: 'MLmetrics'
\end{verbatim}

\begin{verbatim}
## The following object is masked from 'package:base':
## 
##     Recall
\end{verbatim}

\begin{Shaded}
\begin{Highlighting}[]
\FunctionTok{library}\NormalTok{(rmdformats)}
\end{Highlighting}
\end{Shaded}

This is a dataset created to predict the chance of Graduate Admissions.
It was built with the purpose of helping students in shortlisting
universities with accordance to their profiles. The predicted output
gives them a fair idea about their chances for a getting admitted into a
particular univerity.

\begin{Shaded}
\begin{Highlighting}[]
\NormalTok{admission }\OtherTok{\textless{}{-}} \FunctionTok{read.csv}\NormalTok{(}\StringTok{"data\_input/Admission\_Predict.csv"}\NormalTok{) }\SpecialCharTok{\%\textgreater{}\%} 
              \FunctionTok{select}\NormalTok{(}\SpecialCharTok{{-}}\NormalTok{Serial.No.)}

\CommentTok{\# rename nama kolom}
\FunctionTok{names}\NormalTok{(admission) }\OtherTok{\textless{}{-}} \FunctionTok{c}\NormalTok{(}\StringTok{"GRE"}\NormalTok{, }\StringTok{"TOEFL"}\NormalTok{, }\StringTok{"University\_Rating"}\NormalTok{, }\StringTok{"SOP\_Strength"}\NormalTok{, }\StringTok{"LOR\_Strength"}\NormalTok{, }\StringTok{"CGPA"}\NormalTok{, }\StringTok{"Research"}\NormalTok{, }\StringTok{"Admission\_Chance"}\NormalTok{)}
\FunctionTok{head}\NormalTok{(admission)}
\end{Highlighting}
\end{Shaded}

\begin{verbatim}
##   GRE TOEFL University_Rating SOP_Strength LOR_Strength CGPA Research
## 1 337   118                 4          4.5          4.5 9.65        1
## 2 324   107                 4          4.0          4.5 8.87        1
## 3 316   104                 3          3.0          3.5 8.00        1
## 4 322   110                 3          3.5          2.5 8.67        1
## 5 314   103                 2          2.0          3.0 8.21        0
## 6 330   115                 5          4.5          3.0 9.34        1
##   Admission_Chance
## 1             0.92
## 2             0.76
## 3             0.72
## 4             0.80
## 5             0.65
## 6             0.90
\end{verbatim}

This dataset contains several parameters which are considered important
during the application for Masters Programs. The parameters included are
:

\begin{itemize}
\tightlist
\item
  \texttt{GRE}: GRE scores ( out of 340 )
\item
  \texttt{TOEFL}: TOEFL scores ( out of 120 )
\item
  \texttt{University\_Rating}: University rating ( out of 5 )
\item
  \texttt{SOP\_Strength}: Statement of Purpose strength ( out of 5 )
\item
  \texttt{LOR\_Strength} : Letter of Recommendation strength ( out of 5
  )
\item
  \texttt{CGPA} : Undergraduate GPA ( out of 10 )
\item
  \texttt{Research} : Research experience present? ( either 0 or 1 )
\item
  \texttt{Admission\_chance}: Chance of Admit ( ranging from 0 to 1 )
\end{itemize}
\end{frame}

\hypertarget{exploratory-data-analysis}{%
\section{Exploratory Data Analysis}\label{exploratory-data-analysis}}

\begin{frame}[fragile]{Check data structure}
\protect\hypertarget{check-data-structure}{}
\begin{Shaded}
\begin{Highlighting}[]
\FunctionTok{str}\NormalTok{(admission)}
\end{Highlighting}
\end{Shaded}

\begin{verbatim}
## 'data.frame':    400 obs. of  8 variables:
##  $ GRE              : int  337 324 316 322 314 330 321 308 302 323 ...
##  $ TOEFL            : int  118 107 104 110 103 115 109 101 102 108 ...
##  $ University_Rating: int  4 4 3 3 2 5 3 2 1 3 ...
##  $ SOP_Strength     : num  4.5 4 3 3.5 2 4.5 3 3 2 3.5 ...
##  $ LOR_Strength     : num  4.5 4.5 3.5 2.5 3 3 4 4 1.5 3 ...
##  $ CGPA             : num  9.65 8.87 8 8.67 8.21 9.34 8.2 7.9 8 8.6 ...
##  $ Research         : int  1 1 1 1 0 1 1 0 0 0 ...
##  $ Admission_Chance : num  0.92 0.76 0.72 0.8 0.65 0.9 0.75 0.68 0.5 0.45 ...
\end{verbatim}
\end{frame}

\begin{frame}[fragile]{Check data summary}
\protect\hypertarget{check-data-summary}{}
check the summary of your dataset and find anything ``interesting'' from
the summary

\begin{Shaded}
\begin{Highlighting}[]
\FunctionTok{summary}\NormalTok{(admission)}
\end{Highlighting}
\end{Shaded}

\begin{verbatim}
##       GRE            TOEFL       University_Rating  SOP_Strength
##  Min.   :290.0   Min.   : 92.0   Min.   :1.000     Min.   :1.0  
##  1st Qu.:308.0   1st Qu.:103.0   1st Qu.:2.000     1st Qu.:2.5  
##  Median :317.0   Median :107.0   Median :3.000     Median :3.5  
##  Mean   :316.8   Mean   :107.4   Mean   :3.087     Mean   :3.4  
##  3rd Qu.:325.0   3rd Qu.:112.0   3rd Qu.:4.000     3rd Qu.:4.0  
##  Max.   :340.0   Max.   :120.0   Max.   :5.000     Max.   :5.0  
##   LOR_Strength        CGPA          Research      Admission_Chance
##  Min.   :1.000   Min.   :6.800   Min.   :0.0000   Min.   :0.3400  
##  1st Qu.:3.000   1st Qu.:8.170   1st Qu.:0.0000   1st Qu.:0.6400  
##  Median :3.500   Median :8.610   Median :1.0000   Median :0.7300  
##  Mean   :3.453   Mean   :8.599   Mean   :0.5475   Mean   :0.7244  
##  3rd Qu.:4.000   3rd Qu.:9.062   3rd Qu.:1.0000   3rd Qu.:0.8300  
##  Max.   :5.000   Max.   :9.920   Max.   :1.0000   Max.   :0.9700
\end{verbatim}

\textbf{\emph{So far so good. nothing piques my interest}}
\end{frame}

\begin{frame}[fragile]{Check for missing values}
\protect\hypertarget{check-for-missing-values}{}
Make sure that your data is clean from any rows containing any missing
values. If found, rather remove the rows if necessesary

\begin{Shaded}
\begin{Highlighting}[]
\NormalTok{admission }\SpecialCharTok{\%\textgreater{}\%} 
  \FunctionTok{is.na}\NormalTok{() }\SpecialCharTok{\%\textgreater{}\%} 
  \FunctionTok{colSums}\NormalTok{()}\SpecialCharTok{/}\FunctionTok{nrow}\NormalTok{(admission)}
\end{Highlighting}
\end{Shaded}

\begin{verbatim}
##               GRE             TOEFL University_Rating      SOP_Strength 
##                 0                 0                 0                 0 
##      LOR_Strength              CGPA          Research  Admission_Chance 
##                 0                 0                 0                 0
\end{verbatim}

\textbf{\emph{Zero missing values}}
\end{frame}

\begin{frame}[fragile]{Change columns data types}
\protect\hypertarget{change-columns-data-types}{}
From the summary, we can see if University\_Rating and Research to
factor

\begin{Shaded}
\begin{Highlighting}[]
\NormalTok{admission }\OtherTok{\textless{}{-}} 
\NormalTok{  admission }\SpecialCharTok{\%\textgreater{}\%} 
  \FunctionTok{mutate\_at}\NormalTok{(}\FunctionTok{vars}\NormalTok{(University\_Rating, Research), as.factor)}

\FunctionTok{str}\NormalTok{(admission)}
\end{Highlighting}
\end{Shaded}

\begin{verbatim}
## 'data.frame':    400 obs. of  8 variables:
##  $ GRE              : int  337 324 316 322 314 330 321 308 302 323 ...
##  $ TOEFL            : int  118 107 104 110 103 115 109 101 102 108 ...
##  $ University_Rating: Factor w/ 5 levels "1","2","3","4",..: 4 4 3 3 2 5 3 2 1 3 ...
##  $ SOP_Strength     : num  4.5 4 3 3.5 2 4.5 3 3 2 3.5 ...
##  $ LOR_Strength     : num  4.5 4.5 3.5 2.5 3 3 4 4 1.5 3 ...
##  $ CGPA             : num  9.65 8.87 8 8.67 8.21 9.34 8.2 7.9 8 8.6 ...
##  $ Research         : Factor w/ 2 levels "0","1": 2 2 2 2 1 2 2 1 1 1 ...
##  $ Admission_Chance : num  0.92 0.76 0.72 0.8 0.65 0.9 0.75 0.68 0.5 0.45 ...
\end{verbatim}
\end{frame}

\begin{frame}[fragile]{Check for correlation among the columns}
\protect\hypertarget{check-for-correlation-among-the-columns}{}
Majority of the columns are integer, thus one of the exploration to be
done is to check the correlation among the columns.

\begin{Shaded}
\begin{Highlighting}[]
\FunctionTok{ggcorr}\NormalTok{(admission, }\AttributeTok{label =}\NormalTok{ T, }\AttributeTok{hjust =} \FloatTok{0.7}\NormalTok{)}
\end{Highlighting}
\end{Shaded}

\begin{verbatim}
## Warning in ggcorr(admission, label = T, hjust = 0.7): data in column(s)
## 'University_Rating', 'Research' are not numeric and were ignored
\end{verbatim}

\includegraphics{RM_LBB_files/figure-beamer/unnamed-chunk-7-1.pdf}
\textbf{\emph{1. All the numeric columns are positively correlated to
the target column (Admission\_Chance)}} \textbf{\emph{2. CGPA has the
strongest correlation with the target column}} \textbf{\emph{3. GRE has
the second strongest with the target column}}
\end{frame}

\begin{frame}[fragile]{Modeling}
\protect\hypertarget{modeling}{}
Create a linear regression model to predict Admission\_Chance againse
all predictors and store it in modell\_all

\begin{Shaded}
\begin{Highlighting}[]
\NormalTok{model\_all }\OtherTok{\textless{}{-}} \FunctionTok{lm}\NormalTok{(}\AttributeTok{formula =}\NormalTok{ Admission\_Chance }\SpecialCharTok{\textasciitilde{}}\NormalTok{ ., }\AttributeTok{data =}\NormalTok{ admission)}
\end{Highlighting}
\end{Shaded}

Check the summary of the model

\begin{Shaded}
\begin{Highlighting}[]
\FunctionTok{summary}\NormalTok{(model\_all)}
\end{Highlighting}
\end{Shaded}

\begin{verbatim}
## 
## Call:
## lm(formula = Admission_Chance ~ ., data = admission)
## 
## Residuals:
##       Min        1Q    Median        3Q       Max 
## -0.260486 -0.022911  0.009145  0.037471  0.162276 
## 
## Coefficients:
##                      Estimate Std. Error t value Pr(>|t|)    
## (Intercept)        -1.2438075  0.1271202  -9.784  < 2e-16 ***
## GRE                 0.0017118  0.0005986   2.860  0.00447 ** 
## TOEFL               0.0030615  0.0010927   2.802  0.00533 ** 
## University_Rating2 -0.0147251  0.0147760  -0.997  0.31960    
## University_Rating3 -0.0093367  0.0161271  -0.579  0.56296    
## University_Rating4 -0.0073707  0.0196734  -0.375  0.70812    
## University_Rating5  0.0103680  0.0216662   0.479  0.63254    
## SOP_Strength       -0.0026190  0.0055744  -0.470  0.63874    
## LOR_Strength        0.0227892  0.0055456   4.109 4.84e-05 ***
## CGPA                0.1187078  0.0122185   9.715  < 2e-16 ***
## Research1           0.0243631  0.0079708   3.057  0.00239 ** 
## ---
## Signif. codes:  0 '***' 0.001 '**' 0.01 '*' 0.05 '.' 0.1 ' ' 1
## 
## Residual standard error: 0.06375 on 389 degrees of freedom
## Multiple R-squared:  0.8052, Adjusted R-squared:  0.8002 
## F-statistic: 160.8 on 10 and 389 DF,  p-value: < 2.2e-16
\end{verbatim}
\end{frame}

\begin{frame}[fragile]{Feature Selection}
\protect\hypertarget{feature-selection}{}
Choosing all predictos is not the best method to create a regression
model, thus feature selection using stepwise is chosen to choose a set
of the most efficient predictors.

Here I chose \texttt{backward} direction for the stepwise.

\begin{Shaded}
\begin{Highlighting}[]
\NormalTok{model\_back }\OtherTok{\textless{}{-}} \FunctionTok{step}\NormalTok{(model\_all, }\AttributeTok{direction =} \StringTok{"backward"}\NormalTok{)}
\end{Highlighting}
\end{Shaded}

\begin{verbatim}
## Start:  AIC=-2191.44
## Admission_Chance ~ GRE + TOEFL + University_Rating + SOP_Strength + 
##     LOR_Strength + CGPA + Research
## 
##                     Df Sum of Sq    RSS     AIC
## - University_Rating  4   0.01992 1.6006 -2194.4
## - SOP_Strength       1   0.00090 1.5816 -2193.2
## <none>                           1.5807 -2191.4
## - TOEFL              1   0.03190 1.6126 -2185.4
## - GRE                1   0.03323 1.6139 -2185.1
## - Research           1   0.03796 1.6186 -2183.9
## - LOR_Strength       1   0.06862 1.6493 -2176.4
## - CGPA               1   0.38354 1.9642 -2106.6
## 
## Step:  AIC=-2194.43
## Admission_Chance ~ GRE + TOEFL + SOP_Strength + LOR_Strength + 
##     CGPA + Research
## 
##                Df Sum of Sq    RSS     AIC
## - SOP_Strength  1   0.00024 1.6008 -2196.4
## <none>                      1.6006 -2194.4
## - TOEFL         1   0.03291 1.6335 -2188.3
## - GRE           1   0.03585 1.6364 -2187.6
## - Research      1   0.03935 1.6400 -2186.7
## - LOR_Strength  1   0.07445 1.6750 -2178.2
## - CGPA          1   0.41691 2.0175 -2103.8
## 
## Step:  AIC=-2196.38
## Admission_Chance ~ GRE + TOEFL + LOR_Strength + CGPA + Research
## 
##                Df Sum of Sq    RSS     AIC
## <none>                      1.6008 -2196.4
## - TOEFL         1   0.03292 1.6338 -2190.2
## - GRE           1   0.03638 1.6372 -2189.4
## - Research      1   0.03912 1.6400 -2188.7
## - LOR_Strength  1   0.09133 1.6922 -2176.2
## - CGPA          1   0.43201 2.0328 -2102.8
\end{verbatim}

cek summary model

\begin{Shaded}
\begin{Highlighting}[]
\FunctionTok{summary}\NormalTok{(model\_back)}
\end{Highlighting}
\end{Shaded}

\begin{verbatim}
## 
## Call:
## lm(formula = Admission_Chance ~ GRE + TOEFL + LOR_Strength + 
##     CGPA + Research, data = admission)
## 
## Residuals:
##       Min        1Q    Median        3Q       Max 
## -0.263542 -0.023297  0.009879  0.038078  0.159897 
## 
## Coefficients:
##                Estimate Std. Error t value Pr(>|t|)    
## (Intercept)  -1.2984636  0.1172905 -11.070  < 2e-16 ***
## GRE           0.0017820  0.0005955   2.992  0.00294 ** 
## TOEFL         0.0030320  0.0010651   2.847  0.00465 ** 
## LOR_Strength  0.0227762  0.0048039   4.741 2.97e-06 ***
## CGPA          0.1210042  0.0117349  10.312  < 2e-16 ***
## Research1     0.0245769  0.0079203   3.103  0.00205 ** 
## ---
## Signif. codes:  0 '***' 0.001 '**' 0.01 '*' 0.05 '.' 0.1 ' ' 1
## 
## Residual standard error: 0.06374 on 394 degrees of freedom
## Multiple R-squared:  0.8027, Adjusted R-squared:  0.8002 
## F-statistic: 320.6 on 5 and 394 DF,  p-value: < 2.2e-16
\end{verbatim}
\end{frame}

\begin{frame}[fragile]{Evaluasi model}
\protect\hypertarget{evaluasi-model}{}
sekarang anda sudah memiliki 2 model (model\_all, model\_back).
bandingkan kedua model dengan menggunakan fungsi
\texttt{compare\_performance()}

\begin{Shaded}
\begin{Highlighting}[]
\FunctionTok{compare\_performance}\NormalTok{(model\_all, model\_back)}
\end{Highlighting}
\end{Shaded}

\begin{verbatim}
## # Comparison of Model Performance Indices
## 
## Name       | Model |       AIC | AIC weights |       BIC | BIC weights |    R2 | R2 (adj.) |  RMSE | Sigma
## ----------------------------------------------------------------------------------------------------------
## model_all  |    lm | -1054.293 |       0.078 | -1006.396 |     < 0.001 | 0.805 |     0.800 | 0.063 | 0.064
## model_back |    lm | -1059.225 |       0.922 | -1031.284 |       1.000 | 0.803 |     0.800 | 0.063 | 0.064
\end{verbatim}

\textbf{Some Conclusions of the model} 1. From the model summary,
\texttt{GRE}, \texttt{TOEFL} and \texttt{LRE\_Strength} predictors are
the most significant \texttt{state\_pop} dan \texttt{percent\_m} 2. The
adjusted R2 of both models are the same, but the AIC of model\_back is
higher 3. All the predictors have positive impacts towards increasing
the chance of being admited to the university
\end{frame}

\begin{frame}[fragile]{Predict}
\protect\hypertarget{predict}{}
Since \texttt{model\_all} has better AIC compared to
\texttt{model\_back} and both of them have the same adj. R2 and RMSE, I
chose to use \texttt{model\_all}

\textbf{Notes} Pada prakteknya penggunaan data yang sama pada saat
pembuatan model dan pengujian (predict) tidak boleh data yang sama.
minggu depan anda akan mempelajari cara yg lebih tepat.

\begin{Shaded}
\begin{Highlighting}[]
\NormalTok{admission\_pred }\OtherTok{\textless{}{-}} \FunctionTok{predict}\NormalTok{(}\AttributeTok{object =}\NormalTok{ model\_all, }\AttributeTok{newdata =}\NormalTok{ admission )}
\NormalTok{admission\_pred2 }\OtherTok{\textless{}{-}} \FunctionTok{predict}\NormalTok{(}\AttributeTok{object =}\NormalTok{ model\_back, }\AttributeTok{newdata =}\NormalTok{ admission)}
\end{Highlighting}
\end{Shaded}

dari hasil diatas kita bisa mengukur seberapa akurat hasil prediksi yang
didapat dengan mengukur nilai error yang dihasilkan. hitung nilai RMSE
dan MAE dari hasil proses prediksi sebelumnya

\begin{Shaded}
\begin{Highlighting}[]
\FunctionTok{RMSE}\NormalTok{(}\AttributeTok{y\_pred =}\NormalTok{ admission\_pred, }\AttributeTok{y\_true =}\NormalTok{ admission}\SpecialCharTok{$}\NormalTok{Admission\_Chance)}
\end{Highlighting}
\end{Shaded}

\begin{verbatim}
## [1] 0.06286251
\end{verbatim}

\begin{Shaded}
\begin{Highlighting}[]
\FunctionTok{RMSE}\NormalTok{(}\AttributeTok{y\_pred =}\NormalTok{ admission\_pred2, }\AttributeTok{y\_true =}\NormalTok{ admission}\SpecialCharTok{$}\NormalTok{Admission\_Chance)}
\end{Highlighting}
\end{Shaded}

\begin{verbatim}
## [1] 0.06326207
\end{verbatim}

\begin{Shaded}
\begin{Highlighting}[]
\FunctionTok{MAE}\NormalTok{(}\AttributeTok{y\_pred =}\NormalTok{ admission\_pred, }\AttributeTok{y\_true =}\NormalTok{ admission}\SpecialCharTok{$}\NormalTok{Admission\_Chance)}
\end{Highlighting}
\end{Shaded}

\begin{verbatim}
## [1] 0.0451527
\end{verbatim}

\begin{Shaded}
\begin{Highlighting}[]
\FunctionTok{MAPE}\NormalTok{(}\AttributeTok{y\_pred =}\NormalTok{ admission\_pred, }\AttributeTok{y\_true =}\NormalTok{ admission}\SpecialCharTok{$}\NormalTok{Admission\_Chance) }\CommentTok{\# bisa digunakan ketika nilai actual tidak mungkin 0}
\end{Highlighting}
\end{Shaded}

\begin{verbatim}
## [1] 0.07282502
\end{verbatim}

\begin{itemize}
\tightlist
\item
  Nilai RMSE tidak mungkin lebih kecil dari MAE
\item
  RMSE memberikan efek yang lebih besar terhadap outlier, sehingga
  apabila terdapat outlier pada erro akan terdeteksi lebih mudah
\end{itemize}
\end{frame}

\end{document}
